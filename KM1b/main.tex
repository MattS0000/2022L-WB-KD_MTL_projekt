%% Full length research paper template
%% Created by Simon Hengchen and Nilo Pedrazzini for the Journal of Open Humanities Data (https://openhumanitiesdata.metajnl.com)

\documentclass{article}
\usepackage[english]{babel}
\usepackage[utf8]{inputenc}
\usepackage{verbatim}

\title{Title of your full-length research paper}

\author{First Author Name$^{a}$$^{*}$, Second Author Name$^{b}$$^{c}$, etc.$^{a}$$^{c}$ \\
        \small $^{a}$Department, University, City, Country \\
        \small $^{b}$Department, University, City, Country \\
        \small $^{c}$Department, University, City, Country \\\\
        \small $^{*}$Corresponding author: first name, initials, surname; \tt{email.address}
}

\date{} %leave blank

\begin{document}

\maketitle

\begin{abstract} 
\noindent 
\begin{comment}
A short (up to 250 words) summary of the main contributions of the paper and the context of the research. Full length papers discuss and illustrate methods, challenges, and limitations in the creation, collection, management, access, processing, or analysis of data in humanities research, including standards and formats. These aspects must not necessarily be discussed with reference to a specific dataset (or collection thereof) but, if your paper focusses on particular datasets, we advise to add the dataset metadata under the section ‘Dataset description’. This template provides a general outline for full length papers and authors can adapt the headings and include subheadings as they find appropriate. Please delete or replace the blue text with your own text in black. \end{comment}
\end{abstract}

\noindent\keywords{keyword 1; keyword 2; lower case except names, max 6 }\\

\noindent\authorroles{For determining author roles, please use following taxonomy: \url{https://casrai.org/credit/}. Please list the roles for each author.} 

\begin{comment}
\section{Context and motivation}

Describe the context and motivation of your paper.

\subsection{In-text citations}
This journal uses a style based on the APA system (see \href{https://openhumanitiesdata.metajnl.com/about/submissions/#References}{here}). \\
The following are some basic citation commands in \LaTeX: \\

\noindent
\verb|\citet| $\rightarrow$ \citet{jenset&mcgil}\\
\verb|\citet| $\rightarrow$ \citet{shree-a}\\
\verb|\citep| $\rightarrow$ \citep{fabricius-hansen2012b}\\
\verb|\citealp| $\rightarrow$ (\citealp{eckhoff2018a}; \citealp{fabricius-hansen2012b}; \citealp{shree-a})\\










\end{comment}
\section{Related Works}

\begin{itemize}
        \item Overviews of Computer Vision application in the medical field  \cite{OverviewMedical1} and \cite{OverviewMedical2}
        \item Multi Task Learning: first article on the topic  \cite{Multitask1}. Current day overview and a glimpse into the future \cite{Multitask2}.
        \item Survey covering knowledge distilation \cite{KD1} from many different perspectives and approaches.
        \item First papers on combinations of these transfer learning methods appear around 2016. \cite{MTKD2} propose another approach to this problem and \cite{MTKD1} propose usage of those methods in NLP task.
\end{itemize}









\begin{comment}
\section{Dataset description}
Here you can provide, if applicable, information about the dataset(s) whose creation, collection, management, access, processing or analysis have been discussed in this paper, following this schema:
\paragraph{Object name} Typically the name of the file or file set in the repository.
\paragraph{Format names and versions} E.g., ASCII, CSV, Autocad, EPS, JPEG, Excel, SQL, etc.
\paragraph{Creation dates} The start and end dates of when the data was created (YYYY-MM-DD).
\paragraph{Dataset creators} Please list anyone who helped to create the dataset (who may or may not be an author of the data paper), including their roles (using and affiliations).
\paragraph{Language} Languages used in the dataset (i.e., for variable names etc.).
\paragraph{License} The open license under which the data has been deposited (e.g., CC0). 
\paragraph{Repository name} The name of the repository to which the data is uploaded. E.g., Figshare, Dataverse, etc. 
\paragraph{Publication date} If already known, the date in which the dataset was published in the repository (YYYY-MM-DD).

\section{Method}
Describe the methods used in the study.

\section{Results and discussion}
Describe and discuss the results of the study.

\section{Implications/Applications}
Provide information about the implications of this research and/or how it can be applied.


\end{comment}
\bibliographystyle{abbrv}
\bibliography{biblio}

\end{document}